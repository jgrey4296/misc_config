\documentclass[12pt, twoside]{article}
\usepackage[inner=7cm, marginparwidth=3cm, marginparsep=2cm]{geometry}
% \usepackage{marginnote}
\usepackage{bussproofs}
\usepackage{amsmath}
\usepackage{amssymb}
\usepackage[bottom]{footmisc}
\usepackage{sidecap}
\usepackage{graphicx}
\usepackage{latexsym}
\usepackage{fitch}
\usepackage{ND}
\usepackage{cmll}

% This is the "centered" symbol
\def\fCenter{{\mbox{\Large$\rightarrow$}}}
\def\implies{\Large{$\rightarrow$}}

% Optional to turn on the short abbreviations
\EnableBpAbbreviations

% \alwaysRootAtTop  % makes proofs upside down
% \alwaysRootAtBottom % -- this is the default setting
\reversemarginpar
\begin{document}
% \thispagestyle{empty}

This Demonstrates the use of proof tree using bussproof's ``prooftree'' environment,
\marginpar[\footnotesize{with an example of a margin note as well}]{}
a modal logic proof derivation using the ``fitch'' environment, \footnotemark[1]
and Natural Deduction proof using the ``ND'' environment.


\footnotetext[1]{and footnotes}

\vspace*{0.25in}

\section{bussproofs}
A Simple longform proof tree, and then one using the shorthand.

\begin{figure*}[h]
\begin{prooftree}
                \AxiomC{$\Gamma \Delta \vdash A  blah$}
            \UnaryInfC{$B \to C$}
        \RightLabel{Label C}
        \UnaryInfC{C}
        \AxiomC{$\diamond Q$}
    \LeftLabel{Label D}
    \BinaryInfC{D}
    \AxiomC{$E \parr B$}
\BinaryInfC{$\bot$}
\end{prooftree}
\centering
Proof 1
\end{figure*}

\begin{figure*}[h]
\[
  \AxiomC{blah}
  \LL{A}
  \UIC{bloo \quad C}
  \DisplayProof
  \quad
  \AxiomC{aweg}
  \RL{B}
  \UIC{awegjio}
  \DP
\]
\caption{Proof 2} \label{p2}
\end{figure*}

\section{fitch}
Next is a derivation tree.

\begin{equation} \label{eq:f1}
\begin{fitch}
  \fa   blah                         &  testval  \\
  \fh  aweigj                                    \\
  \fa \fp jiaoweg                                \\
  \fs \fr  sub
  % \fa  \fh  bloo                               \\
  % \fa  \fa  \fitchmodalh{blah}  blee           \\
  % \fa  \fa  \fa   aweg                         \\
  % \fa  \fa  \fa   joaiweg                      \\
  % \fa  \fa  \fa   \bot                         \\
\end{fitch}
\end{equation}

\break
\section{nd3}
And finally an nd3 ``ND'' environment proof.

Referring back to ~\eqref{eq:f1}, blah blah blah.


\begin{ND}[Test Proof][testLabel][][][]
  \marginpar{A comment on a proof}
  \ndl{}{test}{obv}\label{1}
  \ndl{\ref{1}}{second}{A}\label{2}
  \ndl{\ref{1},\ref{2}}{third}{B}\label{3}
\end{ND}






\end{document}
